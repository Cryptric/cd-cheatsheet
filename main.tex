\documentclass[8pt, landscape, a4paper]{extarticle}
\usepackage{geometry}[landscape]
\usepackage{multicol}
\usepackage{graphicx}
\usepackage{amsmath} 
\usepackage{amssymb}
\usepackage{ccicons}
\usepackage{hyperref}

\usepackage[dvipsnames]{xcolor}

\usepackage{tikz}
\usepackage{array}

\usepackage{paralist}


\usepackage[compact]{titlesec}

\usepackage{tabularx}
\usepackage{ctable}

\usepackage{listings}
\usepackage{titlesec}

\usepackage{amsthm}
\usepackage[inline]{enumitem}
\usepackage{mdframed}


% reduces section spacing
\titlespacing\section{0pt}{3pt plus 4pt minus 2pt}{0pt plus 2pt minus 2pt}
\titlespacing\subsection{0pt}{2pt plus 4pt minus 2pt}{0pt plus 2pt minus 2pt}

\definecolor{accent}{HTML}{272744}
\definecolor{H1}{HTML}{8b6d9c}
\definecolor{H2}{HTML}{f2d3ab}
\definecolor{H3}{HTML}{494d7e}

\definecolor{dkgreen}{rgb}{0,0.6,0}
\definecolor{gray}{rgb}{0.5,0.5,0.5}
\definecolor{mauve}{rgb}{0.58,0,0.82}

\lstset{frame=,
  language=R,
  aboveskip=0mm,
  belowskip=0mm,
  showstringspaces=false,
  columns=flexible,
  basicstyle={\fontsize{8pt}{9pt}\selectfont\ttfamily},
  numbers=none,
  keywordstyle=\color{blue},
  commentstyle=\color{dkgreen},
  stringstyle=\color{mauve},
  breaklines=true,
  breakatwhitespace=true,
  tabsize=3
}


% Set page margins
\geometry{top=.5cm, left=.5cm, right=.5cm, bottom=.5cm}

% Set indentation
\setlength{\parindent}{0pt}
\setlength{\parskip}{0cm}

% Set path for assets
\graphicspath{{assets/}}

\setlength{\columnsep}{20pt}
\raggedcolumns

% custom environments
\newtheoremstyle{compact_definition}{}{}{\normalfont}{}{\bfseries}{}{0em}{\thmnote{#3}: }
\theoremstyle{compact_definition}
\newtheorem*{definition}{Definition}

% Colored boxes for algorithms
\mdfdefinestyle{mdalgorithm}{
  hidealllines=true,
  innertopmargin=2pt,
  innerbottommargin=2pt,
  innerrightmargin=2pt,
  innerleftmargin=2pt,
  leftmargin=0pt,
  rightmargin=0pt,
  frametitleaboveskip=2pt,
  frametitlebelowskip=2pt,
  theoremseparator={},
  theoremspace={},
  skipabove=2pt,
  skipbelow=2pt,
  backgroundcolor=H1!30,
}

\newenvironment{algorithm}%
    {\begin{mdframed}[style=mdalgorithm]\begin{definition}}%
    {\end{definition}\end{mdframed}}

% _____ CUSTOM COMMANDS __________________________________________
\newcommand{\E}[0]{\mathbb{E}}
\newcommand{\R}[0]{\mathbb{R}}

\newcommand{\sgn}[0]{\text{sgn}}

\newcommand{\argmin}[1]{\underset{#1}{\text{argmin}}}
\newcommand{\argmax}[1]{\underset{#1}{\text{argmax}}}

\renewcommand{\L}{\mathcal{L}}

\begin{document}
\begin{multicols*}{3}

  \fontsize{8pt}{8pt}\selectfont

  \setlength{\abovedisplayskip}{2pt}
  \setlength{\belowdisplayskip}{0pt}
  \setlength{\abovedisplayshortskip}{0pt}
  \setlength{\belowdisplayshortskip}{0pt}

  % _____ CONTENT __________________________________________________

  % main heading
  \begin{center}
    \Large{\textbf{Compiler Design}} \\
    \small{by dcamenisch}
  \end{center}

  %\input{chapters/introduction.tex}
  A compiler translates one programming language to another. The simplified compiler has the following structure:
\begin{compactitem}[$\quad\bullet$]
	\item Lexical Analysis: Source Code $\to$ Token Stream
	\item Parsing: Token Stream $\to$ AST
	\item Intermediate Code Generation: AST $\to$ Intermediate Code
	\item Code Generation: Intermediate Code $\to$ Target Code
\end{compactitem}

The first two steps are the frontend and machine independent, the last step is the backend and machine dependent.

\begin{definition}[Compiler Bug ]
  \begin{itemize*}
      \item Miscompilation (wrong code bug)
      \item Internal compilation error (ICE)
      \item Compiler hang (slow compilation)
      \item Missed optimizations
  \end{itemize*}
\end{definition}

\textbf{Compiler Stages}:
  \begin{compactitem}
    \item Lexing $\rightarrow$ token stream
    \item Parsing $\rightarrow$ abstract syntax
    \item Disambiguation $\rightarrow$ abstract syntax
    \item Semantic analysis $\rightarrow$ annotated abstract syntax
    \item Translation $\rightarrow$ intermediate code
    \item Control-flow analysis $\rightarrow$ control-flow graph
    \item Data-flow analysis $\rightarrow$ interference graph
    \item Register allocation $\rightarrow$ assembly
    \item Code emission
    \item \textbf{Optimization} may be done at many of these stages!
  \end{compactitem}
  

  \section*{x86lite}

x86lite memory consists of $2^{64}$ bytes numbered \texttt{0x00000000} through \texttt{0x0xffffffff}, split into 8-byte quadwords (has to be quadword-aligned). \medskip

The stack grows from high addresses to low addresses, \texttt{rsp} points to the top of the stack, \texttt{rbp} points to the bottom of the current stack frame. \medskip

The stack sits at the top of memory space, at the bottom we have code and data followed by the heap.\medskip

\textbf{Register}: (6 args, 1 ret, ... ({\color{orange}caller saved, \color{blue} callee saved}))
\texttt{\color{orange} rdi, rsi, rdx, rcx, r08, r09, rax, \color{blue} rbx, rsp, rbp, \color{black} rip, \color{orange} r10, r11, \color{blue} r12-r15}\medskip

\textbf{Flags}: \texttt{OF} overflow/underflow, \texttt{SF} sign (1 = negative), \texttt{ZF} zero \medskip

\textbf{Condition Codes}:
\begin{center}
	\begin{tabular}{l l}
		\textbf{Code}         & \textbf{Condition}                            \\
		e (equality)          & \texttt{ZF}                                   \\
		ne (not equals)       & not \texttt{ZF}                               \\
		g (strictly greater)  & not \texttt{ZF} and \texttt{SF} = \texttt{OF} \\
		l (strictly less)     & \texttt{SF} $\neq$ \texttt{OF}                \\
		ge (greater or equal) & \texttt{SF} = \texttt{OF}                     \\
		le (less or equal)    & \texttt{SF} $\neq$ \texttt{OF} or \texttt{ZF} \\
	\end{tabular}
\end{center}

\textbf{Conditional Instructions}:
\begin{center}
	\begin{tabular}{l l}
		\textbf{Instruction}         & \textbf{Description}                            \\
		\texttt{cmpq SRC2, SRC1}          & \texttt{SRC1} - \texttt{SRC2}, set condition flags             \\
        \texttt{setbCC DEST}          & \texttt{DEST}'s lower byte $\leftarrow$ \texttt{if CC then 1 else 0}          \\
        \texttt{jCC SRC}          & \texttt{rip} $\leftarrow$ \texttt{if CC then SRC else fallthrough}             \\
	\end{tabular}
\end{center}

\textbf{Instructions}: \texttt{INSTR} \texttt{SRC} \texttt{DEST} (AT\&T syntax), prefix register with \% and immediate values with \$. Note that \texttt{subq} is \texttt{DEST - SRC}.\medskip

\begin{itemize*}
    \item \textbf{q}uadword: 4 words
    \item \textbf{l}ong: 2 words
    \item \textbf{w}ord: 16 bits
    \item \textbf{b}yte: 8 bits
\end{itemize*} \medskip

\textbf{Operands}:
\begin{compactitem}[$\quad\bullet$]
	\item \texttt{Imm}: 64-bit literal signed integer
	\item \texttt{Lbl}: label representing a machine address
	\item \texttt{Reg}: one of the registers, the value is its content
	\item \texttt{Ind}: machine address
\end{compactitem} \medskip

\texttt{Ind} is  \texttt{offset(base, index)} is calculated \texttt{base + index * 8 + offset}.\medskip

Thus, \texttt{\%rax} refers to the contents of the register, while \texttt{(\%rax)} refers to either the memory address or the contents of the memory address, depending on whether its used as location or value.\medskip

x86 assembly is organized into labeled blocks, indicating code locations used by jumps, etc. Program begins execution at designated label (\texttt{main}).\medskip

\textbf{Calling Conventions}
\begin{compactitem}[$\quad\bullet$]
	\item Setup Stack Frame: \texttt{pushq \%rbp} \quad \texttt{movq \%rsp, \%rbp}

	\item Teardown: \texttt{popq \%rbp}

	\item Caller Save - freely usable by the called code.

	\item Callee Save - must be restored by the called code (\texttt{rbp, rsp, rbx, r12-15}).

	\item Arguments: In \texttt{rdi, rsi, rdx, rcx, r08, r09} and starting with $n = 7$ in $((n-7) + 2) * 8 + \texttt{rbp}$

	\item Return value in \texttt{rax}.

	\item 128 byte "red zone" - scratch pad for the callee (beyond \texttt{rsp}), this means a function can use up to 128 byte without allocationg a stack frame..
\end{compactitem}

  \input{chapters/intermediate.tex}
  \input{chapters/llvm.tex}
  \section*{Lexing}

Lexing is the process of taking the \textbf{source code as an input} and producing a \textbf{token stream as output}. The problem is to precisely define tokens and matching tokens simultaneously.\medskip

One way of implementing a lexer is, using regular expressions. Regex rules precisely describe a sets of strings. But regex alone can be ambiguous if we have multiple matching rules. Most languages therefore choose the longest match or have another specified order. \medskip

Regex can be implemented by forming an NFA and then transforming it to a DFA.

\textbf{Summary Lexer Generator Behavior}
\begin{enumerate}
    \item Take each regular expression $R_i$ and its action $A_i$
    \item Compute the NFA formed by $(R_1 \mid R_2 \mid \cdots \mid R_n)$
    \item Compute the DFA for the big NFA computed in the previous step
    \item Compute the minimal equivalent DFA
    \item Produce the transition table
    \item Implement the longest match
    \begin{enumerate}
        \item Start from the initial state
        \item Follow transitions, remember the last accepted state entered
        \item Accept the input until no transition is possible
        \item Perform the highest-priority action associated with the last accepted state
    \end{enumerate}
\end{enumerate}

\textbf{Regular expressions}
\begin{center}
	\begin{tabular}{l l l l}
		\textbf{Pattern}      & \textbf{Usage}                & \textbf{Pattern}  & \textbf{Usage}  \\
		\texttt{\textquotesingle a \textquotesingle}                   & ordinary char                 & $R_1 \mid R_2$    & $R_1$ or $R_2$      \\
		$R_1R_2$              & concatenation                 & $R*$              & $\geq 0$  repetitions of R \\
		"foo"                 & equal to 'f''o''o'            & $R+$              & $\geq 1$  repetitions of R \\
		$R?$                  & $(\varepsilon \mid R)$        & $['a'-'z']$       & $(a \mid b \mid \dots \mid z)$ \\
		$[$\textbf{\textasciicircum}$'0'-'9']$          & char $\notin \{0, \dots, 9\}$ & R as x            & str matched by R as x \\
	\end{tabular}
\end{center}
  \section*{Parsing}

In this part we take the token stream and generate an \textbf{abstract syntax tree (AST)}. Parsing itself does not check things such as variable scoping, type agreement etc. \medskip

Parsing uses a more powerful tool than regex - \textbf{context free grammars (CFG)}. \medskip

\textbf{Chomsky Hierarchy}:
\begin{compactitem}[$\quad\bullet$]
	\item Regular - Productions have at most one nonterminal and it is at the start or end of the word
	\item Context-Free (CFG) - LHS of productions only have a single nonterminal
	\item Context-Sensitive
	\item Recursively Enumerable
\end{compactitem}

An example for a non CFG would be $a^n b^m c^n d^m$. This corresponds to methods having matching parameters.\medskip

A CFG consists of a set of terminals, a set of nonterminals, a start symbol and a set of productions. A production consists of a single nonterminal LHS and an arbitrary RHS. \medskip

\textbf{Derivation Orders}: Productions can be applied in any order, however they will all lead to the same parse tree. There are two standard orders:
\begin{compactitem}[$\quad\bullet$]
	\item \textbf{Leftmost derivation}: Find the left-most nonterminal and apply a production to it
	\item \textbf{Rightmost derivation}: Find the right-most nonterminal and apply a production there
\end{compactitem}

A grammar is \textbf{ambiguous} iff there are multiple derivation trees for the same word. This can be a problem for associative operators. \medskip

In CFGs ambiguity can (often) be removed by adding nonterminals and allowing recursion only on one side. \smallskip

Derivations for low precedence come first in the grammar. If Left Associative, recursion is on the left side, else on the right side.  \smallskip

For example, we want $+$ to be left associative, $*$ right associative and $*$ has the higher precedence:\smallskip
\begin{lstlisting}
	S -> S + S | S * S | (S) | n
\end{lstlisting} \smallskip

Becomes:\smallskip

\begin{lstlisting}
	S_0 -> S_0 + S_1 | S_1
	S_1 -> S_2 * S_1 | S_2
	S_2 -> n | (S_0)
\end{lstlisting}


\subsection*{LL Grammars and Top-Down Parsing}

When parsing a grammar \textbf{top-down}, we can encounter the problem of multiple productions being possible. LL grammars can \textbf{only} handle right-recursive grammars. \medskip

LL(1) means \textbf{L}eft-to-right scanning, \textbf{L}eft-most derivation, \textbf{1} lookahead symbol. \medskip

Left-factoring a grammar can make it LL(1): If there is a common prefix we can add a new non-terminal at the decision point. We also need to eliminate left-recursion:\smallskip

\begin{lstlisting}
 	S -> S a_1 | ... | S a_n | b_1 | ... | b_m
\end{lstlisting}\smallskip

Becomes:\smallskip

\begin{lstlisting}[escapeinside='']
	S -> b_1 S` | ... | b_m S`		
	S` -> a_1 S` | ... | a_n S` | '$\epsilon$'
\end{lstlisting}\medskip


%------------
%\columnbreak
%------------


To actually use these grammars, we need to translate them into a \textbf{parsing table}: \medskip

For a given production $A \to \gamma$:
\begin{compactitem}[$\quad\bullet$]
	\item Construct the \textbf{first set} of $A$, this set contains all terminals that begin strings derivable from the nonterminal. For each nonterminal of the first set, add the corresponding production to the table.

	\item Construct the \textbf{follow set} of $A$, this set contains all terminals that can appear immediately to the right of the given nonterminal. If $\epsilon$ is derivable by the production, add the corresponding production to the table.
\end{compactitem}

\begin{algorithm}[FOLLOW Set] $\operatorname{FOLLOW}(B)$
\begin{itemize}
    \item For all productions $A \to \alpha B \gamma$, where $\alpha$ and $\gamma$ are arbitrary expressions (might be $\epsilon$)
    \begin{itemize}
        \item[$\bullet$] If $\epsilon \not \in \operatorname{FIRST}(\gamma)$, then $\operatorname{FOLLOW}(B)$ includes all of $\operatorname{FIRST}(\gamma)$
        \item[$\bullet$] If $\epsilon \in \operatorname{FIRST}(\gamma)$, then $\operatorname{FOLLOW}(B)$ includes $(\operatorname{FIRST}(\gamma) \setminus \{ \epsilon \} ) \cup \operatorname{FOLLOW}(A)$
    \end{itemize}
\end{itemize}
    
\end{algorithm}

\begin{center}
	\includegraphics[width=\linewidth]{assets/ll1.png}
\end{center}
\vspace{-10pt}
Intuitively, if we're at nonterminal (\texttt{T, S, S'}) then what productions allow us to parse the terminal (\texttt{+, \$}).

This can be extended to LL($k$) grammars by generating a bigger table.

\begin{algorithm}[LL(1) - Parse Table] $\text{cell}(A, \ldots)$ \\
For each \textbf{non-terminal} $A$ on the $y$-axis of the table do the following:
\begin{itemize}    
    \item $\forall$ \textbf{terminal} $a \in \text{FIRST}(A)$, look at \textbf{all} paths of production rules, through which you can obtain $a$, and add the \textbf{first} production rule of each path, i.e.: $A \mapsto \alpha$, to $\text{cell}(A,a)$ in the table.
    
    \item If $\epsilon \in \text{FIRST}(A)$, then $\forall$ \textbf{terminal} $b \in \operatorname{FOLLOW}(A)$ add $A \mapsto \epsilon$ to $\text{cell}(A,b)$ in the table.
    
\end{itemize}

\end{algorithm}

\textbf{Is it an LL(1) grammar?} No, if there's a cell with multiple options.\medskip

\textbf{LALR(1)}:

Consider for example the following two \textbf{LR(1)} states:

$S1:\{[X \rightarrow \alpha., a], [Y \rightarrow \beta., c]\}, S2:\{[X \rightarrow \alpha., b], [Y \rightarrow \beta., d]\}$

They have same core and can be merged to following LALR(1) state:

$\{[X \rightarrow \alpha., a/b], [Y \rightarrow \beta., c/d]\}$ 

(LALR(1) merges states with same LR(0) rules)

Typically, 10 times fewer LALR(1) states than LR(1). LALR(1) \textbf{may} introduce \textbf{new reduce/reduce} conflicts (but \textbf{no} new shift/reduce conflicts).

\subsection*{LR Grammars and Bottom-Up Parsing}

LR grammars are more expressive than LL grammars. They can handle left-recursive and right-recursive grammars. However error reporting is poorer. \medskip

\textbf{Bottom-up parsing} is a sequence of \textbf{shift} and \textbf{reduce} operations:
\begin{compactitem}[$\quad\bullet$]
	\item Shift: Move look-ahead token to stack.

	\item Reduce: Replace symbols $\gamma$ at the top of the stack with nonterminal $X$ such that $X \to \gamma$ is a production. Pop $\gamma$, push $X$.
\end{compactitem}

The parser state is made up of a stack of nonterminals and terminals, as well as the so far unconsumed input.\medskip

\textbf{Action Selection Problem}:
\begin{compactitem}[$\quad\bullet$]
	\item Given a stack $\sigma$ and a lookahead symbol $b$, should the parser \textbf{shift} $b$ onto the stack (new stack is $\sigma b$) , or \textbf{reduce} a production $X \mapsto \gamma$, assuming that $\sigma = \alpha \gamma$?

	\item Sometimes the parser can reduce, but should not, sometimes the stack can be reduced in different ways.
\end{compactitem}

We want to decide based on a prefix $\alpha$ of the stack and the look-ahead. \medskip

In LR(0) we have states: items to track progress on possible upcoming reductions. An item is a production with an extra separator "." in the RHS. The idea is that the stuff before the "." is already on the stack and the rest is what might be seen next. \medskip

\textbf{Constructing the DFA}:
\begin{compactitem}[$\quad\bullet$]
	\item Add new production: $S' \to S\$$, this is the start of the DFA.

	\item Add all productions whose LHS occurs in an item in the state just after the dot. Note that these items can cause more items to be added until a fixpoint is reached (duplicates allowed).

	\item Add transitions for each possible next (non-)terminal. Shift the dot by one in each of those states.

	\item Every state that ends in a dot is a reduce state.
\end{compactitem}
\vspace{-10pt}
\begin{center}
	\includegraphics[width=0.8\linewidth]{assets/dfa.png}
\end{center}
\vspace{-10pt}

Instead of running the DFA from start for each step, we can store the state with each symbol on the stack - representing the DFA as a table of shape \texttt{state} $\times$ (\texttt{terminals} + \texttt{nonterminals}). \medskip

An LR(0) machine only works if states with reduce actions have a single reduce action else we will encounter shift/reduce or reduce/reduce conflicts (use LR(1) grammar). \medskip

In LR(1), each item is an LR(0) item plus a set of look-ahead symbols $A \to \alpha . \beta, \; \mathcal L$.\medskip

% To form the LR(1) closure, we first do the same as for LR(0). Additionally for each item $C \to . \gamma$ we add due to a rule $A \to \beta . C \gamma, \; \mathcal L$, we compute its look-ahead set $\mathcal M$ including FIRST($\gamma$) and if $\gamma$ can derive $\epsilon$ also $\mathcal L$. \medskip
\begin{algorithm}[LR(1) State closure] Move dot, copy $\L$, apply algorithm
   \begin{itemize}
       \item For each item $A \to \beta . C \delta, \L$
        \begin{itemize}
            \item[$\bullet$] For each token $b \in \operatorname{First}(\delta \L)$
            \begin{itemize}
                \item[$\bullet$] For each production $C \to \gamma$ in Grammar
                \begin{itemize}
                    \item[$\bullet$] Add $C \to .\gamma, b$ to $S$
                \end{itemize}
            \end{itemize}
        \end{itemize}
   \end{itemize}
\end{algorithm}

\vspace{-15pt}
\begin{center}
	\includegraphics[width=0.9\linewidth]{assets/lr1.png}
\end{center}
\vspace{-10pt}
For LR(1) we have a shift-reduce conflict if the shifted token is contained in the follow set of the reduction.

\begin{multicols*}{2}
	\includegraphics[width=1.15\linewidth]{actiondfa.png}
	\includegraphics[width=\linewidth]{language.png}
\end{multicols*}

  \input{chapters/lambda_calculus.tex}
  \input{chapters/typing.tex}
  \input{chapters/compiling_objects.tex}
  \section*{Optimizations}

There are different kinds of optimization: Power, Space, Time.

\begin{compactitem}[$\quad\bullet$]
	\item \textbf{Constant Folding}: If operands are statically known, compute value at compile-time. More general \textbf{algebraic simplification}: Use mathematical identities.

	\item \textbf{Constant Propagation}: If $x$ is a constant replace its uses by the constant.

	\item \textbf{Copy Propagation}: For $x = y$ replace uses of $x$ with $y$

	\item \textbf{Dead Code Elimination}: If side-effect free code can never be observed, safe to eliminate it.

	\item \textbf{Inlining}: Replace a function call with the body of the function (arguments are rewritten to local variables).

	\item \textbf{Code Specialization}: Create Specialized versions of a function that is called form different places with different arguments.

	\item \textbf{Common Subexpression Elimination}: It is the opposite of inlining, fold redundant computations together.

	\item \textbf{Loop Optimizations}
	\begin{compactitem}[$\quad\bullet$]
		\item Hot spots often occur in loops (esp. inner loops)
		\item \textbf{Loop Invariant Code Motion} (hoist outside)
		\item \textbf{Strength Reduction} (replace expensive ops by cheap ones by creating a dependent induction variable)
		\item \textbf{Loop Unrolling}
	\end{compactitem}
\end{compactitem} \medskip

\vspace{-15pt}
\begin{center}
	\includegraphics[width=0.9\linewidth]{assets/optimization_diag.png}
\end{center}
\vspace{-10pt}



\subsection*{Dataflow Analysis}

gen[n] := rhs of the exp (if E = "A = B" then gen[n] = B")\\
kill[n] := all exp containing var on LHS (from the prev. example: A)\medskip

Almost every dataflow analysis is a variation of the following algorithm.\smallskip

\textbf{Forward Must Dataflow Analysis}
\begin{lstlisting}
	for all n, in[n] = T, out[n] = T
	repeat until no change in 'in' or 'out'
		for all n
			in[n] = intersect out[n`] for all n` in pred[n]
			out[n] = gen[n] union (in[n] \ kill[n])	
\end{lstlisting}
\textbf{Backward}: swap \texttt{in} and \texttt{out} and \texttt{pred} with \texttt{succ}.
\textbf{May}: swap $\top$ with $\bot$ or $\emptyset$ and replace \textcolor{blue}{\texttt{intersect}} with \textcolor{blue}{\texttt{union}}.\medskip
For each dataflow analysis we only need to define the set \texttt{gen}, \texttt{kill} as well as the domain of dataflow values $\mathcal L$ and a combining operator $\cup$ or $\cap$.\medskip

Most Data-flow analysis follows this template: \\
$\square_1 \ [n] = \ \blacksquare \ _{n' \in \ \square_3 \ [n]} \ \square_2 \ [n']$ \\
$\square_2 \ [n] = gen[n] \cup (\ \square_1 \ [n] \backslash kill[n])$ \\
And the modifications will be: \\
Forward: $\ \square_1 \ = in \ ; \ \ \ \square_2 = out \ ; \ \ \ \square_3 = pred$ \\
Backward: $\ \square_1 \ = out \ ; \ \ \ \square_2 = in \ ; \ \ \ \square_3 = succ$ \\
May: $\ \blacksquare \ = \bigcup$ \\
Must: $\ \blacksquare \ = \bigcap$ \\
\medskip

\textbf{Liveness} (Backward, May)\medskip

We can use the same registers for multiple \texttt{\%uids} if they are not alive at the same time
(\textbf{live[n]} = uids used before end/reassign).
We define \texttt{gen[s]} as all the variables used (RHS) and \texttt{kill[s]} as all the variables defined by statement $s$ (LHS). $\mathcal L$ corresponds to the variables and the combination operator to the set union.\medskip

It holds: in[$n$] $\supseteq$ gen[$n$], in[$n$] $\supseteq$ out[$n$] $\backslash$ kill[$n$] and out[$n$] $\supseteq$ in[$n'$] if $n' \in $ succ[$n$].\medskip

\textbf{Reaching Definition} (Forward, May)\medskip

What variable definitions reach a particular use of a variable? Used for constant and copy propagation.
\texttt{in / out} is the set of nodes defining some variable such that the definition may reach the beginning resp.
end of the current node. For a statement $d_i: A=B:$ we have $d_i \in \textbf{gen[n]}$ and $\textbf{kill[n]}=\{d_i | d_i \ \text{defines same variable} \land d_i \notin gen[n]\} $\medskip

It holds: out[$n$] $\supseteq$ gen[$n$], in[$n$] $\supseteq$ out[$n'$] if $n' \in$ pred[$n$] and out[$n$] $\supseteq$ in[$n$] $\backslash$ kill[$n$] or out[$n$] $\cup$ kill[$n$] $\supseteq$ in[$n$].\medskip

\textbf{Available Expressions} (Forward, Must)\medskip

Used for common subexpression elimination.
\texttt{in / out} are the set of nodes whose values are available on entry / exit of the current node.
For a statement $d_i: A=B:$ we have $d_i \in \textbf{gen[n]} \;\backslash\; \text{kill[n]}$ and $\textbf{kill[n]}=\{d_i | \text{$d_i$ uses A (RHS)}\}$\medskip

\textbf{Very Busy} (Backward, Must)\medskip

An expression is very busy at location $p$, if every path from $p$ must evaluate the expression before any variable is redefined.
It is used for hoisting expressions. \textbf{in/out/gen allow for expressions (x+y)} \medskip

\texttt{gen[B]} = \{expr; expr a op b is evaluated in B, neither a nor b are subsequently redefinded in B \}\smallskip

\texttt{kill[B]} = \{expr; a or b of expr a op b are defined in B and a op b is not subsequently evaluated in B\} \medskip

\textbf{Dominators} (Forward, Must)\medskip

Define \texttt{dom[n]} as the set of all nodes that dominate \texttt{n}, i.e. \texttt{dom[n] = out[n]}, \texttt{gen} is the singelton set of the node itself, \texttt{kill} is the empty set.\medskip

The iterative solution computes the ideal meet-over-path solution if the flow function distributes over $\cap$. Most of the problems that express properties on how the program computes are distributive and compute the MOP solution, analyses of what the program computes do not (e.g. constant propagation). Our analyses also always terminate, as the flow function (\texttt{out[n] = ...}) is monotonic.\medskip

\textbf{Soundness} is defined as an under approximation of the set of variables.


\subsection*{Register Allocation}

\textbf{Linear-Scan Register Allocation}\medskip

Compute liveness information and then scan through the program, for each instruction try to find an available register, else spill it on the stack.\medskip

\textbf{Graph Coloring}\medskip

Compute liveness information for each temp, create an inference graph (nodes are temps and there is an edge if they are alive at the same time), try to color the graph.\medskip

\begin{algorithm}[Kempe's Algorithm]
\begin{verbatim}
    
1) until all nodes deleted: 
   find a node with degree < k and cut it out of the graph
2) recursively k-color the remaining subgraph, 
   when remaining graph is colored, there must be 
   at least one free color available for the deleted node.
3) if the graph cannot be colored, spill a node and retry.
\end{verbatim}
%\begin{compactitem}[$\quad\bullet$]
%    \item 
%	\item Find a node with degree $< k$ and cut it out of the graph
%	\item Recursively $k$-color the remaining subgraph
%	\item When remaining graph is colored, there must be at least one free color available for the deleted %node.
%	\item If the graph cannot be colored we spill a node and try again.
%\end{compactitem}\medskip
\end{algorithm}

\textbf{Optimistic Coloring}
If we get lucky with the choices of colors made earlier, it is sometimes possible to color a
node marked for spilling.\medskip

\textbf{Accessing Spilled Registers:}
If optimistic coloring fails, we need to generate code to move the spilled temporaries to and from memory
\begin{compactitem}
    \item reserve at least 2 regs for moving to/from memory $\Rightarrow$ 2 less regs available to color graph, but coloring only needed once.
    \item Rewrite the program to use a new temporary with explicit move to and from memory. This allows us to reserve fewer register but introduces a change in live ranges, so we must recompute the liveness and recolor the graph.
\end{compactitem}\medskip

\textbf{Coalescing Interference Graphs}
This can be improved by adding \texttt{move} related edges (temps used in a move should have the same color). More aggresively, we may coalesce two move-related nodes into one. This may increase the degree of a node, so we need to be careful. \medskip

\textbf{Brigg}'s strategy is to only coalesce if the resulting node has fewer than $k$ neighbors with degree $\geq k$. \medskip

\textbf{George}'s strategs is to only coalesce if for every neighbor $t$ of one of the coalescing nodes $x$, $t$ also interferes with the other coalescing node or $t$ has degree $< k$.\medskip

In practice we use George’s strategy if one of x and y is precolored and we use Briggs’ strategy if both are temporaries. \medskip

\textbf{Precolored Nodes}: Certain variables must be pre-assigned to regs (\texttt{call}, \texttt{imul}, caller-save registers) e.g. on X86 imul must define \%rax. To properly allocate temps, we treat registers as nodes in the \textbf{interference graph} with pre-assigned colors. A trick is to treat pre-colored nodes as having
“infinite” degree in the interference graph to guarantee that they won’t be simplified.



\subsection*{Dominator Trees}

To identify loops in a CFG we use domination. $A$ \textbf{dominates} $B$ ($A$ dom $B$), if the only way to reach $B$ from start node is via $A$. This relation is transitive, reflexive and anti-symmetric. This can be computed as forward must dataflow analysis. $A$ \textbf{strictly dominate} $B$, if $A \neq B$ and $A$ dom $B$. The Hasse diagram of the dominates relation is called the \textbf{dominator tree}. bottom-up progress.\medskip

A \textbf{loop} is a set of nodes in the CFG, with a distinguished entry (\textbf{header}) and exit nodes. It is a \textbf{strongly connected component (SSC)}, where every node is reachable from every other node. A loop contains at least 1 \textbf{back edge} (target dominates the source).\medskip

\textbf{How-to: Natural Loop}

For a back edge $s \rightarrow h$, $s=\text{source, }h=\text{header}$:
\begin{compactitem}[$\quad\bullet$]
	\item Look for all nodes dominated by your header (dominates itself)
	\item From these, take the ones which you can use to reach $s$ without going through $h$
	\item Merge loops with the same header $h$
\end{compactitem}\smallskip

We can formally define a loop as:
$$L(s \to h) = \{ n' \; | \; s \text{ is \textbf{reachable} from } n' \text{ in } G
	\backslash\{h\}  \} \cup \{h\}$$

\vspace{3pt}
The \textbf{dominance frontier} of a node $A$ is the set of all CFG nodes
$\gamma$ such that $A$ dominates a predecessor of $\gamma$, but does not
strictly dominate $\gamma$. Intuitively: starting at $A$, there is a path to $\gamma$, but there is another route that does not go through $A$. It is the set of nodes where $A$'s dominance stops. \medskip

\textbf{How-to: Dominance Frontier}

\textbf{preds$[n]$} := direct predecessors in the \textbf{control flow graph (CFG)}, 1st entry can also has \textit{nada} as predecessor, indicated by arrow

\textbf{doms[n]} := direct parent of n in the \textbf{dominator tree}
\begin{algorithm}[dominance frontier algorithm]
\begin{verbatim}

    for B in all nodes:
        1) calc preds[B];
        2) assert |preds[B]| >= 2;
        3) calc doms[B];
        4) start at all preds[B] in dominator tree walk up
           to doms[B] (excluding) and for every node add B 
           to the Dominator frontier    
\end{verbatim}
\end{algorithm}
%For each node $X$: All neighbor nodes it can get to that have some other way to get there
%are part of DF[$X$]. Then do the same for any nodes dominated by $X$ and add them all
%to DF[$X$].\medskip

\textbf{How-to: Least Fixed Point} of \textbf{Join Points} \\

$DF[N] = \cup_{n \in N}DF[n]$ \\
$J[N] = DF_k[N] \text{ where } DF_0 = DF[N];\; DF_{i+1}[N] = DF[DF_i[N] \cup N]$

\begin{minipage}{\columnwidth} % very dirty, but it works to prevent random column break   
	To determine the join points (places where $\phi$-nodes have to be inserted)
	for $N = \{X,Y\}$:
	\begin{compactitem}[$\quad\bullet$]
		\item Find $DF_0[N] = DF[\{X,Y\}] = \{A\} \quad(DF[X] \cup DF[Y] \cup \ldots)$
		\item Find $DF_1[N] = DF[\{X, Y, A\}] = \{Y,A,B\}$
		\item Continue until: $DF_2[N] = DF[\textcolor{blue}{\{X,Y,A,B\}}] = \textcolor{blue}{\{X,Y,A,B\}}$
		\item These are our \textbf{join points}: $J[N] = DF_2[N] = \{X,Y,A,B,C\}$
	\end{compactitem}

\end{minipage}

\subsection*{Single Static Assignment (SSA)}

Each LLVM IR \texttt{\%uid} can be assigned only once. When coming from an \texttt{if-else} branch or similar, we might not know which \texttt{\%uid} to take. That's where we introduce $\phi$-nodes.\medskip

A $\phi$-node picks the version of a variable depending on the label from which the $\phi$-node was entered. It even allows usage of later-defined \texttt{\%uid}s.\medskip

\begin{lstlisting}
	%uid = phi <type> v1, <label1>, ..., vn, <labeln>
\end{lstlisting}\medskip

\textbf{Converting to SSA}:
\begin{compactitem}[$\quad\bullet$]
	\item Start with CFG with \texttt{alloca}s, identify promotable \texttt{alloca}s
	\item Compute dominator tree information
	\item Calculate \texttt{def} / \texttt{use} information for each variable
	\item Insert $\phi$-nodes at necessary join points
	\item Replace \texttt{load} / \texttt{stores} with freshly generated \texttt{\%uid}s
	\item Eliminate unneeded \texttt{alloca}s
\end{compactitem}\medskip


Some \texttt{alloca}s are needed, either if the address of the variable is taken
\begin{verbatim}
%x = alloca i64 // %x cannot be promoted
%y = call malloc(i64 8)
%ptr = bitcast i8* %y to i64**
store i64** %ptr, %x // store the pointer into the heap
\end{verbatim}
or the address escapes by being passed to a function. 
\begin{verbatim}
%x = alloca i64
%y = call foo(i64* %x) // foo may store the pointer into heap
\end{verbatim}
If neither condition holds, it is \textbf{promotable}.\medskip

\textbf{Necessary join points} are defined as the transitive closure of the dominance frontier of all nodes where a variable $x$ is defined or modified. Then we just need to pick the value of $x$ depending on the predecessors of the node where we just inserted the $\phi$-node.\medskip

To place $\phi$-nodes without breaking SSA, we insert \texttt{load}s at the end of each block, and insert \texttt{store}s after $\phi$-nodes. We can then optimize \textbf{load after stores (LAS)} by substituting all uses of the load by the value stored and remove the load itself. Then, we can eliminate dead \texttt{stores} and dead \texttt{alloca}s. At the very end, we can eliminate $\phi$-nodes with only a single value, or identical values from each predecessor.

  \input{chapters/gc.tex}
  \input{chapters/exercises.tex}

  %\begin{center}
  %  \includegraphics[width=0.7\linewidth]{you_can_do_it.jpg}
  %\end{center}

\end{multicols*}
\end{document}

% ____ FOOTER ______________________________________________________
% Content and Template: 
% original by Danny Camenisch (dcamenisch@inf.ethz.ch), 2023
% based on different summaries from many helpful people
