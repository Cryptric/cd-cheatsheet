\section*{x86lite}

x86lite memory consists of $2^{64}$ bytes numbered \texttt{0x00000000} through \texttt{0x0xffffffff}, split into 8-byte quadwords (has to be quadword-aligned). \medskip

The stack grows from high addresses to low addresses, \texttt{rsp} points to the top of the stack, \texttt{rbp} points to the bottom of the current stack frame. \medskip

The stack sits at the top of memory space, at the bottom we have code and data followed by the heap.\medskip

Register: \texttt{rax, rbx, rcx, rdx, rsi, rdi, rsp, rbp, rip, r08-r15}\medskip

Flags: \texttt{OF} overflow/underflow, \texttt{SF} sign (1 = negative), \texttt{ZF} zero \medskip

Condition Codes:
\begin{center}
	\begin{tabular}{l l}
		\textbf{Code}         & \textbf{Condition}                            \\
		e (equality)          & \texttt{ZF}                                   \\
		ne (not equals)       & not \texttt{ZF}                               \\
		g (strictly greater)  & not \texttt{ZF} and \texttt{SF} = \texttt{OF} \\
		l (strictly less)     & \texttt{SF} $\neq$ \texttt{OF}                \\
		ge (greater or equal) & \texttt{SF} = \texttt{OF}                     \\
		le (less or equal)    & \texttt{SF} $\neq$ \texttt{OF} or \texttt{ZF} \\
	\end{tabular}
\end{center}

Instructions: \texttt{INSTR} \texttt{SRC} \texttt{DEST} (AT\&T syntax), prefix register with \% and immediate values with \$. Note that \texttt{subq} is \texttt{DEST - SRC}.\medskip

Operands:
\begin{compactitem}[$\quad\bullet$]
	\item \texttt{Imm}: 64-bit literal signed integer
	\item \texttt{Lbl}: label representing a machine address
	\item \texttt{Reg}: one of the registers, the value is its content
	\item \texttt{Ind}: machine address
\end{compactitem} \medskip

\texttt{Ind} is  \texttt{offset(base, index)} is calculated \texttt{base + index * 8 + offset}.\medskip

Thus, \texttt{\%rax} refers to the contents of the register, while \texttt{(\%rax)} refers to either the memory address or the contents of the memory address, depending on whether its used as location or value.\medskip

x86 assembly is organized into labeled blocks, indicating code locations used by jumps, etc. Program begins execution at designated label (\texttt{main}).\medskip

Calling Conventions
\begin{compactitem}[$\quad\bullet$]
	\item Setup Stack Frame: \texttt{pushq \%rbp} \quad \texttt{movq \%rsp, \%rbp}

	\item Teardown: \texttt{popq \%rbp}

	\item Caller Save - freely usable by the called code.

	\item Callee Save - must be restored by the called code (\texttt{rbp, rsp, rbx, r12-15}).

	\item Arguments: In \texttt{rdi, rsi, rdx, rcx, r08, r09} and starting with $n = 7$ in $(n-7) + 2) * 8 + \texttt{rbp}$

	\item Return value in \texttt{rax}.

	\item 128 byte "red zone" - scratch pad for the callee (beyond \texttt{rsp}), this means a function can use up to 128 byte without allocationg a stack frame..
\end{compactitem}
