\section*{LLVM (Low Level Virtual Machine)}

\textbf{Storage Types}: local variable \texttt{\%uid}, global variable \texttt{@gid}, abstract locations (stack-allocated with \texttt{alloca}), heap-allocated structures (\texttt{malloc}).\medskip

Each \texttt{\%uid} appears on the left-hand side of an assignment only once in the entire control flow graph (SSA).\medskip

The entry block of the CFG does not have to be labeled, the last instruction of a block is called the terminator. \medskip

\textbf{Datatypes:}
\begin{compactitem}
\item $<elts>$ is an integer constant $\geq 0$
\item Structure types can be named at the top level: such structure types can be recursive
\end{compactitem}
\begin{lstlisting}
t ::=
    void
    i1 | i8 | i64 # N-bit integers
    [<elts> x t] # arrays
    fty # function types
    {t1, t2, ..., tn} # structures
    t* # pointers
    %Tident # named types
    
fty ::= # function types
    t (t1, ..., tn) # return, argument types

%T1 = type {t1, t2, ..., tn} # named type
\end{lstlisting} \medskip

\textbf{Example Program:}
\begin{lstlisting}
@s = global i32 42

declare void @use (i64)

define i64 @foo(i64 %a, i64* %b) { 
	%sum = add nsw i64 %a, 42
	%cond = icmp sgt i64 %sum, 100
	br i1 %cond , label %then , label %else
	
then:
	call void @use(i64 %sum) 
	ret i64 %sum
	
else:
	store i64 %sum, i64* %b 
	ret i64 %sum
}
\end{lstlisting} \medskip

\texttt{nsw}: informs the compiler, that the operation before the \texttt{nsw}, will not cause a signed overflow

\textbf{SSA}: every variable can only be once on the left-hand side of an assignment and is immutable (unchangeable) after that assignment. If a variable has multiple assignments in the original code, it gets split into different "versions", each with a unique name when converted to the \textbf{SSA}-form.

\subsection*{GEP}

LLVM supports structured data with the use of \texttt{types}, e.g.:\medskip

\begin{lstlisting}
	%struct.Node = type {i64, %struct.Node*}
\end{lstlisting}\medskip

To compute pointer values of structs or index into arrays, LLVM provides the \texttt{getelementptr} instruction. Given a pointer and a path through the structured data pointed to by that pointer, GEP computes an address - analog of LEA.\medskip

\begin{lstlisting}
	getelementptr <ty>* <ptrval> {, <ty> <idx>}* 
\end{lstlisting}\medskip

GEP never dereferences the address it is calculating.
